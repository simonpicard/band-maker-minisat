\documentclass[11pt]{article}
\usepackage[utf8]{inputenc}

\usepackage{fullpage,epsf,fancyheadings}
\usepackage{amsmath}
\usepackage{graphicx}
\usepackage{tikz}

\newcommand{\vrai}{\mbox{\sc Vrai}}
\newcommand{\faux}{\mbox{\sc Faux}}
\newcommand{\prop}{proposition}
\newcommand{\imp}{\rightarrow}
\newcommand{\lorb}{\stackrel{}{\bar{\lor}}}
\newcommand{\landb}{\stackrel{}{\bar{\land}}}
\newcommand{\Caml}{Caml Light}
\newcommand{\C}[1]{{\cal C}(#1)}
\newcommand{\question}{?}

\newtheorem{defi}{Définition}
\reversemarginpar

\begin{document}

	\begin{titlepage}
	\begin{tikzpicture}[remember picture, overlay]
	  \node [anchor=north east, inner sep=0pt]  at (current page.north east)
	     {\includegraphics[height=3cm]{Baniere_ULB.png}};
	\end{tikzpicture}
	\begin{center}
	\textbf{\textsc{UNIVERSIT\'E LIBRE DE BRUXELLES}}\\
	\textbf{\textsc{Faculté des Sciences}}\\
	\textbf{\textsc{Département d'Informatique}}
	\vfill{}\vfill{}
	\begin{center}{\Huge INFO-F-302 Informatique Fondamentale : Rapport de projet}\end{center}{\Huge \par}
	\begin{center}{\large \textsc{Omer} Nicolas \\\textsc{Picard} Simon}\end{center}{\Huge \par}
	\vfill{}\vfill{}
	\vfill{}\vfill{}\enlargethispage{3cm}
	
	\begin{figure} [h!]
             \centering
	    \includegraphics[width=4cm]{Sigle_ULB.png}
	\end{figure}
	
	\textbf{Année académique 2013~-~2014}
	\end{center}
	
	\end{titlepage}
	
\tableofcontents
\pagebreak

\section{Introduction}

Le but du projet d'informatique fondamentale est de former des groupes de musique sous certaines contraintes. Il s'agit d'exprimer les contraintes sous forme normale conjonctive afin de les encoder dans l'outil permettant de résoudre les différents problèmes donnés. Pour cela, nous utilisons le solveur \textit{MiniSat} qui permet de déterminer, sous un ensemble de contraintes, si un problème est satisfaisable.

\section{Question 1}

\subsection{Définition des variables et symboles}

Dans cette section, nous définissons les différentes variables et symboles que nous utilisons dans l'expression des contraintes sous forme normale conjonctive.

\begin{itemize}
\item M est le nombre de musiciens.
\item I est le nombre d'instruments.
\item K est le nombre maximum de groupes.
\item $X_{a,b,c}$ est une variable telle que pour tout musicien \textit{a}, l'instrument \textit{b} dans le groupe \textit{c}, $X_{a,b,c}$ = vrai si et seulement si le musicien \textit{a} joue de l’instrument \textit{b} dans le groupe \textit{c}.
\item $I_a$ représente la liste des instrument que \textit{a} maitrise.
\end{itemize}

\subsection{Forme Normale Conjonctive}

\subsubsection{Contrainte d'existence}

$$\bigwedge\limits_{0<a\le M}\ \left(\bigvee\limits_{0<c\le K}\ \bigvee\limits_{b\in I_a}\ X_{a,b,c}\right)$$
Il s'agit simplement de dire que chaque musicien doit jouer un des instruments qu'il maitrise dans un groupe.

\subsubsection{Un musicien ne peut être que dans un seul groupe}

$$\bigwedge\limits_{0<a\le M}\ \bigwedge\limits_{b_1\in I_a}\ \bigwedge\limits_{b_2\in I_a}\ \bigwedge\limits_{0<c_1\le K}\ \bigwedge\limits_{c_1<c_2\le K}\ \left(\neg X_{a,b_1,c_1} \vee \neg X_{a,b_2,c_2}\right)$$

Par musicien, nous disons qu'il ne peut pas, à la fois jouer d'un instrument $b_1$ quelconque dans un groupe $c_1$ et un instrument $b_2$ quelconque dans un autre groupe $c_2$.

\subsubsection{Un instrument par groupe}

$$\bigwedge\limits_{0<a\le M}\ \bigwedge\limits_{0<c\le K}\ \bigwedge\limits_{b_1\in I_a}\ \bigwedge\limits_{\substack{b_2\in I_a \\ b_2 \ne b1}}\ \left(\neg X_{a,b_1,c} \vee \neg X_{a,b_2,c}\right)$$

Par musicien, nous disons qu'il ne peut pas à la fois jouer d'un instrument $b_1$ dans un groupe c et d'un autre instrument $b_2$ quelconque dans le même groupe c.

\subsubsection{Un seul musicien peut jouer d'un certain instrument dans un groupe}

$$\bigwedge\limits_{0<c\le K}\ \bigwedge\limits_{0<a_1\le M}\ \bigwedge\limits_{a_1<a_2\le M}\ \bigwedge\limits_{\substack{b\in I_{a_1} \\ b\in I_{a_2}}}\ \left(\neg X_{a_1,b,c} \vee \neg X_{a_2,b,c}\right)$$

Nous exprimons ici que, dans un même groupe, une personne $a_1$ et autre $a_2$ ne peuvent pas jouer du même instrument dans un groupe.

\subsubsection{Tout les instruments dans un groupe ou aucun}

$$\bigwedge\limits_{0<c\le K}\ \bigwedge\limits_{0<a_1\le M}\ \bigwedge\limits_{b_1\in I_{a_1}}\ \bigwedge\limits_{\substack{0<b_2\le I \\ b_2 \ne b_1}}\ \left(\neg X_{a_1,b_1,c} \bigvee\limits_{\substack{0<a_2\le M \\ a_2 \ne a_1 \\ b_2 \in I{a_2}}}\ X_{a_2,b_2,c}\right)$$

Ici nous voulons exprimer la contrainte groupe complet ou vide, pour ce faire nous utilisons une implication qui dit que, dans chaque groupe, si un musicien $a_1$ joue d'un instrument $b_1$, alors ceci implique que, pour chaque autre instrument $b_2$, un des autres musiciens maîtrisant ce dernier, $a_2$, le joue dans ce groupe.

\section{Question 2}

\subsection{Définition des variables et symboles}

\begin{itemize}
\item $Max_a$ représente le nombre maximum de groupe dans lequel le musicien \textit{a} peut jouer.
\item $|I_a|$ représente la cardinalité de $I_a$ soit le nombre d'instrument dont \textit{a} peut se servir.
\end{itemize}

Les variables et symboles restants sont identiques à ceux de la question 1.

\subsection{Forme Normale Conjonctive}

\subsubsection{Un musicien ne peut pas jouer dans plus de $Max_a$ groupe}

Pour implémenter cela, nous allons étendre la solution "un groupe maximum" qui, pour rappel, voulait exprimer $\neg (groupe_1 \wedge groupe_2) \wedge \neg (groupe_2 \wedge groupe_3) \ldots$
Dans la conjonction se trouveront $Max_a+1$ littéraux. Pour se faire, on commence par générer tous les sous-ensembles de groupe de taille $Max_a+1$ et toutes les combinaisons de $I_a$ de taille $Max_a+1$. Car un joueur ne peut pas jouer d'un même instrument dans différents groupes ni de différents instruments dans différents groupes. 

A la seconde étape, on génère toutes les combinaisons d'identifiants ($t$ et $s$) et finalement, il ne reste qu'à choisir une des combinaisons avec les variables $x$ et $y$.

Cette contrainte est exprimée par la première ligne de la forme normale conjonctive suivante. On peut illustrer cela par un exemple où $K$ = 3 et $Max_a+1$ = 2

$$\bigwedge\limits_{0<a\le M}\ \bigwedge\limits_{0<c_1\le K}\ \bigwedge\limits_{c_1<c_2\le K}\ \ldots\ \bigwedge\limits_{c_{{Max_a}}<c_{Max_{a}+1}\le K}\ \bigwedge\limits_{b_1 \in I_a}\ \bigwedge\limits_{b_2 \in I_a}\ \ldots\  \bigwedge\limits_{b_{Max_a+1} \in I_a} $$

$$ \bigwedge\limits_{0<t_{1}\le Max_a+1}\  \ldots\  \bigwedge\limits_{0<t_{Max_a+1}\le Max_a+1}\ \bigwedge\limits_{0<s_1\le Max_a+1}\  \ldots\  \bigwedge\limits_{0<s_{Max_a+1}\le Max_a+1}\\$$

$$ \bigwedge\limits_{0<y\le Max_a+1}\  \left( \bigvee\limits_{0<z\le Max_a+1}\  \neg X_{a,b_{s_{y}},c_{t_{z}}}\right)$$

Nous nous rendons compte qu'avec cette contrainte, nous vérifions aussi la contrainte \textsl{Un instrument par groupe} puisque parfois, pour tout $y$, $s_y$ sera identique pour toute la clause, dès lors nous n'avons plus besoin de cette contrainte.\\
Pour le reste, les autres contraintes sont identiques.\\

Le nombre de clauses crée sera égal à : \\

$$\sum_{a=1}^{M}\ \left( {K \choose Max_a+1}\ * \ |I_a|^{Max_a+1}\ *\ \left(Max_a+1\right)^{Max_a+1}\ *\ \left(Max_a+1\right)^{Max_a+1}\ *\  \left(Max_a+1\right) \right) $$

Soit\\

$$\sum_{a=1}^{M}\ \left( {K \choose Max_a+1}\ * \ |I_a|^{Max_a+1}\ *\ \left(Max_a+1\right)^{2*(Max_a+1)+1}\ \right) $$

Sachant que chaque clause est composée des $Max_a$ littéraux. 
Cependant, il y aura des doublons dans les clauses générées mais le solveur MiniSat se chargera de les supprimer pour nous\footnote{addendum au nombre de clauses créé à la section \ref{addendum}}.

\section{Question 3}

\subsection{Définition des variables}

\begin{itemize}
\item L'instrument qui a comme valeur \textit{I} est toujours l'instrument \textsl{chant}.
\end{itemize}

Les variables et les symboles sont identiques à ceux de la question 2.

\subsection{Forme Normale Conjonctive}

\subsubsection{Un seul musicien peut jouer d'un certain instrument dans un groupe}

C'est toujours vrai sauf quand l'instrument considéré est le chant.

$$\bigwedge\limits_{0<c\le K}\ \bigwedge\limits_{0<a_1\le M}\ \bigwedge\limits_{a_1<a_2\le M}\ \bigwedge\limits_{\substack{b\in I_{a_1} \\ b\in I_{a_2} \\ b \ne I}}\ \left(\neg X_{a_1,b,c} \vee \neg X_{a_2,b,c}\right)$$

\subsubsection{Tout les instruments dans un groupe ou aucun}

La contrainte reste vraie, cependant si un musicien utilise un autre instrument que sa voix, ceci peut impliquer qu'il chante et inversement (s'il maîtrise ces instruments bien sur).\\

$$\bigwedge\limits_{0<c\le K}\ \bigwedge\limits_{0<a_1\le M}\ \bigwedge\limits_{b_1\in I_{a_1}}\ \bigwedge\limits_{\substack{0<b_2\le I \\ b_2 \ne b_1}}\ \left(\neg X_{a_1,b_1,c} \ \bigvee\limits_{\substack{0<a_2\le M \\ b_2 \in I{a_2}}}\left( \
\bigvee\limits_{\substack{a_2 \ne a_1 \\ b_1 \ne I}}\  X_{a_2,b_2,c}\ 
\bigvee\limits_{\substack{b_1 = I}}\ X_{a_2,b_2,c}\ 
\bigvee\limits_{\substack{a_2 = a_1 \\ b_2 = I}}\ X_{a_2,b_2,c}\right)\right)$$

Dans la parenthèse la plus imbriquée, le premier grand ou est le même que précédemment, le second signifie que, si $a_1$ chante, alors $a_2$ peut chanter aussi et le dernière représente la fait que si $a_1$ joue d'un instrument et qu'il sait chanter, il peut faire les deux.

\subsubsection{Un instrument par groupe}

La contrainte \textsl{Un instrument par groupe} et la contrainte \textsl{Un musicien ne peut pas jouer dans plus de $Max_a$ groupes} ne peuvent plus être combinées, puisque la première n'est plus toujours vraie depuis l'arrivée de la composante du chant.\\

Commençons par réintroduire cette dernière :

$$\bigwedge\limits_{0<a\le M}\ \bigwedge\limits_{0<c\le K}\ \bigwedge\limits_{\substack{b_1\in I_a \\ b_1 \ne I}}\ \bigwedge\limits_{\substack{b_2\in I_a \\ b_2 \ne b1 \ne I}}\ \left(\neg X_{a,b_1,c} \vee \neg X_{a,b_2,c}\right)$$

Il s'agit de la même expression que dans la première question sauf que $b_1$ et $b_2$ ne peuvent pas valoir $I$, le \textsl{chant}.

\subsubsection{Un musicien ne peut pas jouer dans plus de $Max_a$ groupe}

Cette contrainte reste vraie, sauf pour l'instrument $I$ et il ne faut plus introduire en même temps la contrainte décrite ci dessus. Voici la forme normale conjonctive :\\

$$\bigwedge\limits_{0<a\le M}\ \bigwedge\limits_{0<c_1\le K}\ \bigwedge\limits_{c_1<c_2\le K}\ \ldots\ \bigwedge\limits_{c_{{Max_a}}<c_{Max_{a}+1}\le K}\ \bigwedge\limits_{\substack{b_1 \in I_a \\ b_1 \ne I}}\ \bigwedge\limits_{\substack{b_2 \in I_a \\ b_2 \ne I}}\ \ldots\  \bigwedge\limits_{\substack{b_{Max_a+1} \in I_a \\ b_{Max_a} \ne I}} $$

$$ \bigwedge\limits_{0<t_{1}\le Max_a+1}\  \ldots\  \bigwedge\limits_{t_{Max_a}<t_{Max_a+1}\le Max_a+1}\ \bigwedge\limits_{0<s_1\le Max_a+1}\  \ldots\  \bigwedge\limits_{0<s_{Max_a+1}\le Max_a+1}\\$$

$$ \bigwedge\limits_{0<y\le Max_a+1}\  \left( \bigvee\limits_{0<z\le Max_a+1}\  \neg X_{a,b_{s_{y}},c_{t_{z}}}\right)$$

Ici, nous ne prenons plus tous les ensembles d'indices de $c$, $t$ mais uniquement les ensembles différents, nous nous retrouvons alors avec un coefficient binomial ${Max_a \choose Max_a}$ qui vaut 1, en effet il s'agit simplement de la suite des indices ce qui nous donne bien ce qu'on souhaite, chaque $c_t$ sera différent des autres.

\label{addendum}

Dans ce cas, les clauses générées auront pour but unique d'introduire une contrainte \textsl{au plus k} et il y en aura :

$$\sum_{a=1}^{M}\ \left( {K \choose Max_a+1}\ * \ |I_a|^{Max_a+1}\ *\ \left(Max_a+1\right)\ *\ \left(Max_a+1\right)^{Max_a+1}\ *\  \left(Max_a+1\right) \right) $$

Soit\\

$$\sum_{a=1}^{M}\ \left( {K \choose Max_a+1}\ * \ |I_a|^{Max_a+1}\ *\ \left(Max_a+1\right)^{Max_a+3}\ \right) $$

Il est important de noter qu'on omettra l'instrument \textsl{chant} dans l'ensemble $I_a$.

\end{document}