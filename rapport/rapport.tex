\documentclass[11pt]{article}
\usepackage[utf8]{inputenc}

\usepackage{fullpage,epsf,fancyheadings}
\usepackage{french}
\usepackage{amsmath}
\usepackage{graphicx}
\usepackage{tikz}

\newcommand{\vrai}{\mbox{\sc Vrai}}
\newcommand{\faux}{\mbox{\sc Faux}}
\newcommand{\prop}{proposition}
\newcommand{\imp}{\rightarrow}
\newcommand{\lorb}{\stackrel{}{\bar{\lor}}}
\newcommand{\landb}{\stackrel{}{\bar{\land}}}
\newcommand{\Caml}{Caml Light}
\newcommand{\C}[1]{{\cal C}(#1)}
%\newcommand{\question}{\marginpar{\hfill$\Box$}}
\newcommand{\question}{?}

\newtheorem{defi}{Définition}
\reversemarginpar


\begin{document}

	\begin{titlepage}
	\begin{tikzpicture}[remember picture, overlay]
	  \node [anchor=north east, inner sep=0pt]  at (current page.north east)
	     {\includegraphics[height=3cm]{Baniere_ULB.png}};
	\end{tikzpicture}
	\begin{center}
	\textbf{\textsc{UNIVERSIT\'E LIBRE DE BRUXELLES}}\\
	\textbf{\textsc{Faculté des Sciences}}\\
	\textbf{\textsc{Département d'Informatique}}
	\vfill{}\vfill{}
	\begin{center}{\Huge INFO-F-302 Informatique Fondamentale : Rapport de projet}\end{center}{\Huge \par}
	\begin{center}{\large \textsc{Omer} Nicolas \\\textsc{Picard} Simon}\end{center}{\Huge \par}
	\vfill{}\vfill{}
	\vfill{}\vfill{}\enlargethispage{3cm}
	
	\begin{figure} [h!]
             \centering
	    \includegraphics[width=4cm]{Sigle_ULB.png}
	\end{figure}
	
	\textbf{Année académique 2013~-~2014}
	\end{center}
	
	\end{titlepage}
	
    \pagebreak

\section{Question 1}

\subsection{Forme Normale Conjonctive}

\subsubsection{Définition des variables}

\begin{itemize}
\item M est le nombre de musiciens.
\item I est le nombre d'instruments.
\item K est le nombre maximum de groupes.
\item $X_{a,b,c}$ est une variable qui défini si le musicien \textit{a} jour l'insrtuement \textit{b} dans le groupe \textit{c}.
\item $I_a$ représente la liste des instrument que \textit{a} maitrise.
\end{itemize}

\subsubsection{Contrainte d'existence}

$$\bigwedge\limits_{0<a\le M}\ \left(\bigvee\limits_{0<c\le K}\ \bigvee\limits_{b\in I_a}\ X_{a,b,c}\right)$$

\subsubsection{Un musicien ne peut être que dans un seul groupe}

$$\bigwedge\limits_{0<a\le m}\ \bigwedge\limits_{b_1\in I_a}\ \bigwedge\limits_{b_2\in I_a}\ \bigwedge\limits_{0<c_1\le K}\ \bigwedge\limits_{c_1<c_2\le K}\ \left(\neg X_{a,b_1,c_1} \vee \neg X_{a,b_2,c_2}\right)$$

\subsubsection{Un instrument par groupe}

$$\bigwedge\limits_{0<a\le M}\ \bigwedge\limits_{0<c\le K}\ \bigwedge\limits_{b_1\in I_a}\ \bigwedge\limits_{\substack{b_2\in I_a \\ b_2 \ne b1}}\ \left(\neg X_{a,b_1,c} \vee \neg X_{a,b_2,c}\right)$$

\subsubsection{Un seul musicien peut jouer d'un certain instrument dans un groupe}

$$\bigwedge\limits_{0<c\le K}\ \bigwedge\limits_{0<a_1\le M}\ \bigwedge\limits_{a_1<a_2\le M}\ \bigwedge\limits_{\substack{b\in I_{a_1} \\ b\in I_{a_2}}}\ \left(\neg X_{a_1,b,c} \vee \neg X_{a_2,b,c}\right)$$

\subsubsection{Tout les instruments dans un groupe ou aucun}

$$\bigwedge\limits_{0<c\le K}\ \bigwedge\limits_{0<a_1\le M}\ \bigwedge\limits_{b_1\in I_{a_1}}\ \bigwedge\limits_{\substack{0<b_2\le I \\ b_2 \ne b_1}}\ \left(\neg X_{a_1,b_1,c} \bigvee\limits_{\substack{0<a_2\le M \\ a_2 \ne a_1 \\ b_2 \in I{a_2}}}\ \neg X_{a_2,b,c}\right)$$

\section{Question 2}

\section{Question 3}

\end{document}