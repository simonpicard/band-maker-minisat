\documentclass[11pt]{article}
\usepackage[utf8]{inputenc}

\usepackage{fullpage,epsf,fancyheadings}
\usepackage{french}
\usepackage{amsmath}
\usepackage{graphicx}
\usepackage{tikz}

\newcommand{\vrai}{\mbox{\sc Vrai}}
\newcommand{\faux}{\mbox{\sc Faux}}
\newcommand{\prop}{proposition}
\newcommand{\imp}{\rightarrow}
\newcommand{\lorb}{\stackrel{}{\bar{\lor}}}
\newcommand{\landb}{\stackrel{}{\bar{\land}}}
\newcommand{\Caml}{Caml Light}
\newcommand{\C}[1]{{\cal C}(#1)}
%\newcommand{\question}{\marginpar{\hfill$\Box$}}
\newcommand{\question}{?}

\newtheorem{defi}{Définition}
\reversemarginpar


\begin{document}

	\begin{titlepage}
	\begin{tikzpicture}[remember picture, overlay]
	  \node [anchor=north east, inner sep=0pt]  at (current page.north east)
	     {\includegraphics[height=3cm]{Baniere_ULB.png}};
	\end{tikzpicture}
	\begin{center}
	\textbf{\textsc{UNIVERSIT\'E LIBRE DE BRUXELLES}}\\
	\textbf{\textsc{Faculté des Sciences}}\\
	\textbf{\textsc{Département d'Informatique}}
	\vfill{}\vfill{}
	\begin{center}{\Huge INFO-F-302 Informatique Fondamentale : Rapport de projet}\end{center}{\Huge \par}
	\begin{center}{\large \textsc{Omer} Nicolas \\\textsc{Picard} Simon}\end{center}{\Huge \par}
	\vfill{}\vfill{}
	\vfill{}\vfill{}\enlargethispage{3cm}
	
	\begin{figure} [h!]
             \centering
	    \includegraphics[width=4cm]{Sigle_ULB.png}
	\end{figure}
	
	\textbf{Année académique 2013~-~2014}
	\end{center}
	
	\end{titlepage}
	
    \pagebreak

\section{Introduction}

Le but du projet d'informatique fondamentale est de former des groupes de musique sous certaines contraintes. Il s'agit d'exprimer les contraintes sous forme normale conjonctive afin de les encoder dans l'outil permettant de résoudre les différents problèmes donnés. Pour cela, nous utilisons le solveur \textit{MiniSat} qui permet de déterminer, sous un ensemble de contraintes, si un problème est satisfaisable.

\section{Question 1}

\subsection{Forme Normale Conjonctive}

\subsubsection{Définition des variables}

Dans cette section, nous définissons les différentes variables que nous utilisons dans l'expression des contraintes sous forme normale conjonctive.

pour tout musicien a, instrument b et groupe c, Xa,b,c = vrai ssi le musicien a joue de l’instrument b dans le groupe c.

\begin{itemize}
\item M est le nombre de musiciens.
\item I est le nombre d'instruments.
\item K est le nombre maximum de groupes.
\item $X_{a,b,c}$ est une variable telle que pour tout musicien \textit{a}, l'instrument \textit{b} dans le groupe \textit{c}, $X_{a,b,c}$ = vrai si et seulement si le musicien \textit{a} joue de l’instrument \textit{b} dans le groupe \textit{c}.
\item $I_a$ représente la liste des instrument que \textit{a} maitrise.
\end{itemize}

\subsubsection{Contrainte d'existence}

$$\bigwedge\limits_{0<a\le M}\ \left(\bigvee\limits_{0<c\le K}\ \bigvee\limits_{b\in I_a}\ X_{a,b,c}\right)$$
Il s'agit simplement de dire que chaque musicien doit jouer un des instruments qu'il maitrise dans un groupe.

\subsubsection{Un musicien ne peut être que dans un seul groupe}

$$\bigwedge\limits_{0<a\le M}\ \bigwedge\limits_{b_1\in I_a}\ \bigwedge\limits_{b_2\in I_a}\ \bigwedge\limits_{0<c_1\le K}\ \bigwedge\limits_{c_1<c_2\le K}\ \left(\neg X_{a,b_1,c_1} \vee \neg X_{a,b_2,c_2}\right)$$

Par muscien, nous disons qu'il ne peut pas, à la fois jouer d'un instrument $b_1$ quelconque dans un groupe $c_1$ et un instrument $b_2$ quelconque dans un autre groupe $c_2$.

\subsubsection{Un instrument par groupe}

$$\bigwedge\limits_{0<a\le M}\ \bigwedge\limits_{0<c\le K}\ \bigwedge\limits_{b_1\in I_a}\ \bigwedge\limits_{\substack{b_2\in I_a \\ b_2 \ne b1}}\ \left(\neg X_{a,b_1,c} \vee \neg X_{a,b_2,c}\right)$$

Par muscien, nous disons qu'il ne peut pas, à la fois jouer d'un instrument $b_1$ dans un groupe c et d'un autre instrument $b_2$ quelconque dans le même groupe c.

\subsubsection{Un seul musicien peut jouer d'un certain instrument dans un groupe}

$$\bigwedge\limits_{0<c\le K}\ \bigwedge\limits_{0<a_1\le M}\ \bigwedge\limits_{a_1<a_2\le M}\ \bigwedge\limits_{\substack{b\in I_{a_1} \\ b\in I_{a_2}}}\ \left(\neg X_{a_1,b,c} \vee \neg X_{a_2,b,c}\right)$$

Nous exprimons ici que, dans un même groupe, une personne $a_1$ et autre $a_2$ ne peuve pas jouer du même instrument.

\subsubsection{Tout les instruments dans un groupe ou aucun}

$$\bigwedge\limits_{0<c\le K}\ \bigwedge\limits_{0<a_1\le M}\ \bigwedge\limits_{b_1\in I_{a_1}}\ \bigwedge\limits_{\substack{0<b_2\le I \\ b_2 \ne b_1}}\ \left(\neg X_{a_1,b_1,c} \bigvee\limits_{\substack{0<a_2\le M \\ a_2 \ne a_1 \\ b_2 \in I{a_2}}}\ \neg X_{a_2,b,c}\right)$$

Ici nous voulons exprimer la contrainte groupe complet ou vide, pour ce faire nous utilisons une implication qui dit que, dans chaque groupe, si un musicien $a_1$ joue d'un instrument $b_1$, alors ceci implique que, pour chaque autre instrument $b_2$, un des autres musiciens maîtrisant ce dernier, $a_2$, le joue dans ce groupe.

\section{Question 2}

\begin{itemize}
\item $Max_a$ représente le nombre maximum de groupe dans lequel le musicien \textit{a} peut jouer.
\end{itemize}

\subsubsection{Contrainte d'existence}

\subsubsection{Un musicien ne peut pas jouer dans plus de $Max_a$ groupe}

$$\bigwedge\limits_{0<a\le M}\ \bigwedge\limits_{0<c_1\le K}\ \bigwedge\limits_{c_1<c_2\le K}\ \ldots\ \bigwedge\limits_{c_{{Max_a}-1}<c_{Max_a}\le K}\ \bigwedge\limits_{b_1 \in I_a}\ \bigwedge\limits_{b_2 \in I_a}\ \ldots\  \bigwedge\limits_{b_{Max_a} \in I_a} $$

$$\bigwedge\limits_{0<n_{k_0}\le K}\  \ldots\  \bigwedge\limits_{n_{k_{Max_a-1}}<n_{k_{Max_a}}\le K}\ \bigwedge\limits_{0<n_{i_0}\le I}\  \ldots\  \bigwedge\limits_{0<n_{i_{Max_a}}\le I}$$

$$ \bigwedge\limits_{0<y\le Max_a}\  \left( \bigvee\limits_{0<z\le Max_a}\  \neg X_{a,b_{n_{i_{y}}},c_{n_{k_{z}}}}\right)$$

Nous nous rendons qu'avec cette contrainte, nous vérifions aussi "Un instrument par groupe" puisque parfois $n_i$ sera le même pour toute la clause, dès lors nous n'avons plus besoin de cette contrainte.\\
Pour les reste, les autres contraintes sont identiques.\\

Le nombre de clauses crée sera : \\

$$\sum_{a=1}^{M}\ \left( {K \choose Max_a}\ * \ I^{Max_a}\ *\ {K \choose Max_a}\ *\ I^{Max_a}\ *\  Max_a \right) $$

Sachant que chaque clauses est composé des $Max_a$ litéraux.


\section{Question 3}

\end{document}